% type
\documentclass{article}

% format
\usepackage[letterpaper, margin=1.5cm]{geometry}
\usepackage[utf8]{inputenc}
% font
\renewcommand{\familydefault}{\sfdefault}
\usepackage[dvipsnames]{xcolor}
\usepackage{sectsty}
\allsectionsfont{\color{DarkOrchid}}

% language
\usepackage[spanish]{babel}

% math symbols
\usepackage{amssymb}

% macros
\newcommand{\tx}[1]{\texttt{#1}}

% code
\usepackage{minted}

% header

\title {
    Práctica 6\\
    Excepciones y Continuaciones en la Máquina $\mathcal{K}$
}

\author {
    Sandra del Mar Soto Corderi \quad Edgar Quiroz Castañeda
}

% document
\begin{document}
    \maketitle

    \section{La Maquina $\mathcal{K}$}

    \subsection{Marcos}

    Para modelar la máquina abstracta $\mathcal{K}$, en necesario tener una 
    estrucutra que guarde los cómputos pendientes, llamados marcos. Hay un marco
    por cada posible cómputo pendiente de algún operador.

    \begin{minted}{haskell}
    -- | Tipo de marcos vacíos
    type Pending = ()

    -- | Tipo de marco
    data Frame = SuccF Pending
               | PredF Pending
               | ...
               | RaiseF Pending
               | HandleF Pending Identifier Expr
               | ContinueFL Pending Expr
               | ContinueFR Expr Pending
               deriving (Eq)
    \end{minted}

    \begin{itemize}
        \item \tx{instance Show Frame}

        Sólo se tomó un el inicio del nombre del operador correspondiente, 
        remplazando el cómputo pendiente con un guión \tx{-}.

        \begin{minted}{haskell}
        -- | Show para marcos
        instance Show Frame where
            show ex =
            case ex of
                (SuccF _) -> "suc(-)"
                (PredF _) -> "pred(-)"
                ...
                (RaiseF _) -> "raise(-)"
                (HandleF _, x, e2) -> "handle(-, " ++ (show x) ++ ", " ++ (show e2) ++ ")"
                (ContinueFL _ e) -> "continue(-, " ++ (show e) ++ ")"
                (ContinueFR e _) -> "continue(" ++ (show e) ++ ", -)"
        \end{minted}
    \end{itemize}

    \subsection{Estados}

    Ahora que hay que tener un registro de los cómputos pendientes, los estados
    al momento de evaluar expresiones cambian.

    Hay tres posibles estados: evaluando, terminando, y error.

    \begin{minted}{haskell}
        -- | Tipo para estados
        data State = E (Memory, Stack, Expr) | R (Memory, Stack, Expr) 
              | P (Memory, Stack, Expr)
    \end{minted}

    \begin{itemize}
        \item \tx{instance Show State}

        En los ejemplos en la descripción de la práctica, los estados se
        mostraban directamente como están definidos.

        Así que para mantener este formato de manera sencilla, sólo se agregó 
        que los estados derivaran su instancia de \tx{Show} de su definición.

        \begin{minted}{haskell}
            -- | Tipo para estados
            data State = E (Memory, Stack, Expr) | R (Memory, Stack, Expr) 
              | P (Memory, Stack, Expr)
              deriving (Show)
        \end{minted}
    \end{itemize}

    \section{Excepciones y Continuaciones}

    \begin{itemize}
        \item \tx{eval1}

        Para esta función, había que traducir la función equivalente anterior 
        que se usó para evaluar expresiones usando memoria.

        Hay diferentes casos por estado.

        \begin{itemize}
            \item \tx{E (Memory, Stack, Expr)}

            Para estas expresiones, si ya se llegó a un valor en todas las 
            expresiones necesarioas para determinar el operador, entonces se 
            pasa al estado \tx{R (Memory, Stack, Expr)} con el valor obtenido 
            por el operador.

            Si aún falta algo por evaluar, entonces se bota el marco actual y se
            mete el marco necesario para seguir con la evaluación.

            \begin{minted}{haskell}
                eval1 (E (m, s, e)) = 
                    case e of
                      -- valores
                      (V _) -> R (m, s, e)
                      ...
                      -- operadores unarios
                      (Succ e1) -> E (m, (SuccF ()):s, e1)
                      ...
                      -- operadores binarios
                      (Add e1 e2) -> E (m, (AddFL () e2):s, e1)
                      ...
                      -- otros
                      (If e1 e2 e3) -> E (m, (IfF () e2 e3):s, e1)
                      (Let x e1 e2) -> E (m, (LetF x () e2):s, e1)
                      (Letcc x e1) -> E (m, s, subst e1 (x, Cont s))
                      -- error
                      _ -> P (m, s, e)
            \end{minted}

            \item \tx{R (Memory, Stack, Expr)}

            Se tiene que regresar un valor. Dependiendo del valor y del cómputo
            pendiente en el tope de la pila de marcos, se realiza alguna
            operación.

            Como evaluar la expresión resultante, devolver un valor, o manejar 
            la memoria.

            Ests operaciones fueron una traducción casi directa de las definias 
            en la versión anterior de \tx{evals}.

            Si no se puede realizar ninguna operación, se propaga un error.

            Se muestran algunos ejemplos de transición de este estado.

            \begin{minted}{haskell}
            eval1 (R (mem, s, e)) =
                case e of
                ...
                (I m) ->
                    case s of
                        ((SuccF _) : s') -> R (mem, s', I (succ m))
                        ...
                        ((ContinueFR (Cont s'') _) : s') -> R (mem, s'', e)
                        _ -> P (mem, s, Raise e)
                (L i) ->
                    ...
                    ((AssigFL _ e2) : s') -> E (mem, ((AssigFR e ()) : s'), e2)
                    ((AssigFR (L i) _):s') -> 
                        case update (i, e) mem of
                            Just mem' -> R (mem', s', Void)
                            Nothing -> P (mem, s, Raise e)
                        ...
                (Void) ->
                    case s of 
                        ...
                        ((SeqF _ e2) : s') -> E (mem, s', e2)
                        ...
                (Cont st) ->
                    case s of
                        ...
                        ((ContinueFL _ e2):s') -> E (mem, ((ContinueFR e ()):s'), e2)
                        ((ContinueFR (Cont s'') _) : s') -> R (mem, s'', e)
                        _ -> P (mem, s, Raise e)
                ...
            \end{minted}

            Es la sección más larga de la función.

            \item \tx{P (Memory, Stack, Expr)}

            Hay dos casos. 
            
            Si no se está lidiando con errores, entonces sólo se
            sacan todos los valores de la pila de marcos hasta llegar al marco
            vacío. Llegar a este punto significa un error en tiempo de 
            ejecución.

            Si se está lidiando con errores (usando \tx{handle}), entonces se 
            desenvuelve el valor de error y se para a la expresión que se usará
            para recuperarse del error.

            \begin{minted}{haskell}
                eval1 (P (mem, s, e)) =
                    case s of
                        -- lidiando con errores
                        (HandleF _ x e1):s' ->
                            case e of 
                            (Raise e1) -> E (mem, s', subst e1 (x, e1))
                            _ -> P (mem, s, Raise e)
                        -- dejando pasar errores
                        (_:s') -> P (mem, s', e)
            \end{minted}

        \end{itemize}
        \item\tx{evals}

        El propósito de esta función es dar pasos en la evaluación usando 
        \tx{eval1} hasta llegar a un estado de fin de evaluación 
        \tx{R (mem, [], v)} o un estado de error \tx{P (mem, [], e)}.

        En general, hay que vaciar la pila de marcos. Así que simplemente hay 
        que evaluar el estado hasta que la pila de marcos esté vacía y la 
        expresión esté bloqueada.

        \begin{minted}{haskell}
            -- | Evaular un estado exhaustivamente
            evals :: State -> State
            evals s = 
                case eval1 s of
                s'@(E (_, [], e')) -> if blocked e' then s' else evals s'
                s'@(E (_, _, e')) -> evals s'
                s'@(R (_, [], e')) -> if blocked e' then s' else evals s'
                s'@(R (_, _, e')) -> evals s'
                s'@(P (_, [], e')) -> if blocked e' then s' else evals s'
                s'@(P (_, _, e')) -> evals s'
        \end{minted}
        \item \tx{eval}

        Utiliza \tx{evals} para evaluar lo más posible una expresión. Si es un 
        valor, se regresa. Si es otra cosa, se lanza un error.

        \begin{minted}{haskell}
        eval :: Expr -> Expr
        eval e = 
            case evals (E ([], [], e)) of
            R (_, [], e') -> 
                case e' of
                B _ -> e'
                I _ -> e'
                _ -> error "invalid final value"
            _ -> error "no value was returned"
        \end{minted}
    \end{itemize}

    \section{Integración}

    Se tuvo que hacer modificaciónes al módulo \tx{Sintax} para incluir las 
    expresiones concernientes a errores y continuaciones.

    \begin{minted}{haskell}
        -- | Tipo para las expresiones
        data Expr = ...
                    | Raise Expr
                    | Handle Expr Identifier Expr
                    | Letcc Identifier Expr
                    | Continue Expr Expr
                    | Cont Stack
                    | Error
                    deriving (Eq)
    \end{minted}
    \begin{itemize}
        \item \tx{frVars}
        \begin{minted}{haskell}
-- | Obteniendo variables libres de una expresion(AddFL _ e) -> "add(-, " ++ (show e) ++ ")"
frVars :: Expr -> [Identifier]
frVars ex =
	case ex of
		...
		(Void) -> []
		(Seq e f) -> union (frVars e) (frVars f)
		(While e f) -> union (frVars e) (frVars f)
		(Raise e) -> frVars e
		(Handle e i f) -> union (frVars e) ((frVars f) \\ [i])
		(LetCC i e) -> (frVars e) \\ [i]
		(Continue e f) -> union (frVars e) (frVars f)
		(Cont s) -> []
        \end{minted}
        
        
        \item \tx{subst}
        \begin{minted}{haskell}
    -- | Aplicando substitucion si es semanticamente posible
subst :: Expr -> Substitution -> Expr
subst ex s@(y, e') =
	case ex of
		...
		(Seq e f) -> Seq (st e) (st f)
		(While e f) -> While (st e) (st f)
		(Raise e) -> Raise (st e)
		(Handle e x f) ->
			if x == y || elem x (frVars e')
				then st (alphaExpr ex)
			else Handle e x (st f)
		(LetCC x e) ->
			if x == y || elem x (frVars e')
				then st (alphaExpr ex)
			else LetCC x (st e)
		(Continue e f) -> Continue (st e) (st f)
		(Cont a) -> Cont a
		where st = (flip subst) s
        \end{minted}
        
        \item \tx{alphaEq}
        \begin{minted}{haskell}
-- | Dice si dos expresiones son alpha equivalentes
alphaEq :: Expr -> Expr -> Bool
...
alphaEq (Seq e1 e2) (Seq f1 f2) = (alphaEq e1 f1) && (alphaEq e2 f2)
alphaEq (While e1 e2) (While f1 f2) = (alphaEq e1 f1) && (alphaEq e2 f2)
alphaEq (Raise e) (Raise f) = alphaEq e f
alphaEq (Handle e1 x e2) (Handle f1 y f2) = (alphaEq e1 f1) && (alphaEq e2 (subst f2 (y, (V x))))
alphaEq (LetCC x e) (LetCC y f) = (alphaEq e (subst f (y, (V x))))
alphaEq (Continue e1 e2) (Continue f1 f2) = (alphaEq e1 f1) && (alphaEq e2 f2)
alphaEq (Cont s1) (Cont s2) = s1 == s2
alphaEq _ _ = False
        \end{minted}
    \end{itemize}

    \section*{Dificultades}
    Nos fue difícil la implementación del eval1 ya que algunas definiciones nos causaban confusión y la longitud del código era bastante larga, por lo que era fácil que hubieran errores en el código.Igual fue complicado manejar los casos que manipulaban las continuaciones y la memoria a la vez.
    
\end{document}


